% LISA User Manual
%
% Please note this document will be automatically compiled and hosted online
% after each commit to master. Because of this, renaming or moving the
% document should be done carefully. To see the compiled document, go to
% https://exosports.github.io/LISA/doc/LISA_User_Manual.html

\documentclass[letterpaper, 12pt]{article}
\textwidth=6.5in
\textheight=9.5in
\topmargin=-0.75in
\oddsidemargin=0.0in
\evensidemargin=0.0in

\usepackage{graphicx}
\usepackage{enumitem}
\usepackage{amssymb, amsmath}
\usepackage{xcolor}
\usepackage{listings}

\usepackage{times}
\usepackage{natbib}
\usepackage{etoolbox}
\usepackage{astjnlabbrev-jh}
\usepackage{bibentry}
\usepackage{ifthen}
\usepackage{epsfig}

\usepackage{commath}
\usepackage{rotating}

\usepackage{dirtree}
\usepackage{changepage}
\usepackage{alltt}

%\usepackage[T1]{fontenc}
%\usepackage[scaled]{beramono}
%\usepackage[utf8]{inputenc}
%\renewcommand*\familydefault{\ttdefault}

% \lstset{
% language=Python,
% showstringspaces=false,
% formfeed=\newpage,
% tabsize=4,
% commentstyle=\itshape,
% basicstyle=\ttfamily,
% morekeywords={models, lambda, forms}
% }
 
% \newcommand{\code}[2]{
% \hrulefill
% \subsection*{#1}
% \lstinputlisting{#2}
% \vspace{2em}
% }

% Default fixed font does not support bold face
\DeclareFixedFont{\ttb}{T1}{txtt}{bx}{n}{10} % for bold
\DeclareFixedFont{\ttm}{T1}{txtt}{m}{n}{10}  % for normal
% twelve-sized ttm:
\DeclareFixedFont{\tttb}{T1}{txtt}{bx}{n}{12}  % for bold
\DeclareFixedFont{\tttm}{T1}{txtt}{m} {n}{12}  % for normal

\DeclareFixedFont{\ttnm}{T1}{txtt}{m}{n}{9.8}  % for normal

% Custom colors
\usepackage{color}
\definecolor{deepblue}  {rgb}{0.0, 0.0, 0.5}
\definecolor{deepred}   {rgb}{0.8, 0.0, 0.0}
\definecolor{deepgreen} {rgb}{0.0, 0.5, 0.0}
\definecolor{commentc}  {rgb}{0.5, 0.5, 0.5}
\definecolor{DodgerBlue}{rgb}{0.1, 0.6, 1.0}

% Python style for highlighting
\newcommand\pythonstyle{\lstset{
language=Python,
basicstyle = \ttm,
morekeywords = {self, as, assert, with, yield}, % Add keywords here
keywordstyle  = \ttb\color{blue},      %
emph        = {MyClass, __init__},     % Custom highlighting
emphstyle   = \ttb\color{DodgerBlue},  % Custom  highlighting style
stringstyle = \color{deepred},         % Strings highlighting style
commentstyle=\color{commentc},         % Comment highlighting style
frame       = tb,                      % Any extra options here
showstringspaces = false
}}

% Python environment:
\lstnewenvironment{python}[1][]{\pythonstyle\lstset{#1}}{}
% Python for external files:
\newcommand\pythonexternal[2][]{{\pythonstyle\lstinputlisting[#1]{#2}}}
% Python for inline:
\newcommand\pythoninline[1]{{\pythonstyle\lstinline!#1!}}

% Python style for highlighting
\newcommand\plainstyle{\lstset{
language=Python,
basicstyle = \ttnm,
keywordstyle  = \ttnm,      %
emph        = {MyClass, __init__},     % Custom highlighting
emphstyle   = \ttnm\color{black},    % Custom  highlighting style
stringstyle = \color{black},         % Strings highlighting style
commentstyle=\color{black},         % Comment highlighting style
frame       = tb,                      % Any extra options here
showstringspaces = false
}}

% Plain environment:
\lstnewenvironment{plain}[1][]{\plainstyle\lstset{#1}}{\vspace{10pt}}
\newcommand\plaininline[1]{{\plainstyle\lstinline!#1!}}

% To use boldface verbatim:
%\lstset{basicstyle=\ttfamily,
%        escapeinside={||},
%        mathescape=true}

\lstset{
    language={[LaTeX]TeX},
    basicstyle=\tt\color{red},
    escapeinside={||},
}

\bibliographystyle{apj_hyperref}
\usepackage[%pdftex,      %%% hyper-references for pdflatex
bookmarks=true,           %%% generate bookmarks ...
bookmarksnumbered=true,   %%% ... with numbers
colorlinks=true,          % links are colored
citecolor=blue,           % green   % color of cite links
linkcolor=blue,           %cyan,         % color of hyperref links
menucolor=blue,           % color of Acrobat Reader menu buttons
urlcolor=blue,            % color of page of \url{...}
breaklinks=true,
linkbordercolor={0 0 1},  %%% blue frames around links
pdfborder={0 0 1},
frenchlinks=true]{hyperref}
%\usepackage{breakurl}

\newcommand{\eprint}[1]{\href{http://arxiv.org/abs/#1}{#1}}
\newcommand{\ISBN}[1]{\href{http://cosmologist.info/ISBN/#1}{ISBN: #1}}
\providecommand{\adsurl}[1]{\href{#1}{ADS}}

% hyper ref only the year in citations:
\makeatletter
% Patch case where name and year are separated by aysep:
\patchcmd{\NAT@citex}
  {\@citea\NAT@hyper@{%
     \NAT@nmfmt{\NAT@nm}%
     \hyper@natlinkbreak{\NAT@aysep\NAT@spacechar}{\@citeb\@extra@b@citeb}%
     \NAT@date}}
  {\@citea\NAT@nmfmt{\NAT@nm}%
   \NAT@aysep\NAT@spacechar\NAT@hyper@{\NAT@date}}{}{}
% Patch case where name and year are separated by opening bracket:
\patchcmd{\NAT@citex}
  {\@citea\NAT@hyper@{%
     \NAT@nmfmt{\NAT@nm}%
     \hyper@natlinkbreak{\NAT@spacechar\NAT@@open\if*#1*\else#1\NAT@spacechar\fi}%
       {\@citeb\@extra@b@citeb}%
     \NAT@date}}
  {\@citea\NAT@nmfmt{\NAT@nm}%
   \NAT@spacechar\NAT@@open\if*#1*\else#1\NAT@spacechar\fi\NAT@hyper@{\NAT@date}}
  {}{}
\makeatother


%\def\bibAnnoteFile#1{}
%\bibpunct[, ]{(}{)}{,}{a}{}{,}

% Packed reference list:
\setlength\bibsep{0pt}

% \pagestyle{myheadings}
% \markright{MC\sp{3}}
% \pagenumbering{arabic}


% :::::::::::::::::::::::
\newcommand\degree{\degr}
\newcommand\degrees{\degree}
\newcommand\vs{\emph{vs.}}

% unslanted mu, for ``micro'' abbrev.
\DeclareSymbolFont{UPM}{U}{eur}{m}{n}
\DeclareMathSymbol{\umu}{0}{UPM}{"16}
\let\oldumu=\umu
\renewcommand\umu{\ifmmode\oldumu\else\math{\oldumu}\fi}
\newcommand\micro{\umu}
\newcommand\micron{\micro m}
\newcommand\microns{\micron}

\let\oldsim=\sim
\renewcommand\sim{\ifmmode\oldsim\else\math{\oldsim}\fi}
\let\oldpm=\pm
\renewcommand\pm{\ifmmode\oldpm\else\math{\oldpm}\fi}
\newcommand\by{\ifmmode\times\else\math{\times}\fi}
\newcommand\ttt[1]{10\sp{#1}}
\newcommand\tttt[1]{\by\ttt{#1}}
\newcommand\tablebox[1]{\begin{tabular}[t]{@{}l@{}}#1\end{tabular}}
\newbox{\wdbox}
\renewcommand\c{\setbox\wdbox=\hbox{,}\hspace{\wd\wdbox}}
\renewcommand\i{\setbox\wdbox=\hbox{i}\hspace{\wd\wdbox}}
\newcommand\n{\hspace{0.5em}}
\newcommand\marnote[1]{\marginpar{\raggedright\tiny\ttfamily\baselineskip=9pt #1}}
\newcommand\herenote[1]{{\bfseries #1}\typeout{======================> note on page \arabic{page} <====================}}
\newcommand\fillin{\herenote{fill in}}
\newcommand\fillref{\herenote{ref}}
\newcommand\findme[1]{\herenote{(FINDME: #1)}}

\newcount\timect
\newcount\hourct
\newcount\minct
\newcommand\now{\timect=\time \divide\timect by 60
         \hourct=\timect \multiply\hourct by 60
         \minct=\time \advance\minct by -\hourct
         \number\timect:\ifnum \minct < 10 0\fi\number\minct}

\newcommand\citeauthyear[1]{\citeauthor{#1} \citeyear{#1}}

\newcommand\mc{\multicolumn}
\newcommand\mctc{\multicolumn{2}{c}}


% {\tttm -h, --help} \\
% Print the list of arguments. \newline

% \newenvironment{myindentpar}[1]%
%   {\begin{list}{}%
%          {\setlength{\leftmargin}{3cm}}%
%      \item[]%
%   }
% {\end{list}}

\newenvironment{packed_enum}{
\begin{enumerate}[leftmargin=3cm]
   \setlength{\itemsep}{1pt}
   \setlength{\parskip}{5pt}
   \setlength{\parsep}{0pt}
}{\end{enumerate}}

\newcommand{\argument}[2]{{\noindent\tttm #1}%
\begin{adjustwidth}{2.5em}{0pt}%
#2 \vspace{0.3cm}%
\end{adjustwidth}%
}

%\newcommand{\ttmb}[1]{\tttm\color{#1}}
\newcommand{\routine}[2]{{\noindent\tttm\color{blue} #1:}%
\begin{adjustwidth}{2.0em}{0pt}%
#2 \vspace{0.15cm}%
\end{adjustwidth}%
}

% \newcommand{\routine}[2]{{\noindent\tttm\color{blue} #1}%
% \begin{adjustwidth}{0.0em}{0pt}%
%  #2 \vspace{0.15cm}%
% \end{adjustwidth}%
% }

% :::::::::::::::::: jhmacs2.tex :::::::::::::::::::::::::::::::::::::
\typeout{Joe Harrington's personal setup, Wed Jun 17 10:53:17 EDT 1998}
% Tue Mar 29 22:23:03 EST 1994

% :::::: pato.tex ::::::
% Joetex character unreservations.
% This file frees most of TeX's reserved characters, and provides
% several alternatives for their functions.


% utility
\catcode`@=11

% comments are first....
\newcommand\comment[1]{}

\newcommand\commenton{\catcode`\%=14}
\newcommand\commentoff{\catcode`\%=12}

% Not-a-comment:
\newcommand\nocomment[1]{#1}

\renewcommand\math[1]{$#1$}
\newcommand\mathshifton{\catcode`\$=3}
\newcommand\mathshiftoff{\catcode`\$=12}

\comment{alignment tab}
\let\atab=&
\newcommand\atabon{\catcode`\&=4}
\newcommand\ataboff{\catcode`\&=12}

\let\oldmsp=\sp
\let\oldmsb=\sb
\def\sp#1{\ifmmode
           \oldmsp{#1}%
         \else\strut\raise.85ex\hbox{\scriptsize #1}\fi}
\def\sb#1{\ifmmode
           \oldmsb{#1}%
         \else\strut\raise-.54ex\hbox{\scriptsize #1}\fi}
\newbox\@sp
\newbox\@sb
\def\sbp#1#2{\ifmmode%
           \oldmsb{#1}\oldmsp{#2}%
         \else
           \setbox\@sb=\hbox{\sb{#1}}%
           \setbox\@sp=\hbox{\sp{#2}}%
           \rlap{\copy\@sb}\copy\@sp
           \ifdim \wd\@sb >\wd\@sp
             \hskip -\wd\@sp \hskip \wd\@sb
           \fi
        \fi}
\def\msp#1{\ifmmode
           \oldmsp{#1}
         \else \math{\oldmsp{#1}}\fi}
\def\msb#1{\ifmmode
           \oldmsb{#1}
         \else \math{\oldmsb{#1}}\fi}
\def\supon{\catcode`\^=7}
\def\supoff{\catcode`\^=12}
\def\subon{\catcode`\_=8}
\def\suboff{\catcode`\_=12}
\def\supsubon{\supon \subon}
\def\supsuboff{\supoff \suboff}


\newcommand\actcharon{\catcode`\~=13}
\newcommand\actcharoff{\catcode`\~=12}

\newcommand\paramon{\catcode`\#=6}
\newcommand\paramoff{\catcode`\#=12}

\comment{And now to turn us totally on and off...}

\newcommand\reservedcharson{ \commenton  \mathshifton  \atabon  \supsubon
                             \actcharon  \paramon}

\newcommand\reservedcharsoff{\commentoff \mathshiftoff \ataboff \supsuboff
                             \actcharoff \paramoff}

\newcommand\nojoe[1]{\reservedcharson #1 \reservedcharsoff}

\catcode`@=12
\reservedcharsoff

\reservedcharson
\newcommand\jhauth[1]{{#1}}
\newcommand\jhstud[1]{{#1}}

\comment{Must have ONLY ONE of these... trust these macros, they work
\newcommand\jhauth[1]{{\bfseries #1}}
\newcommand\jhstud[1]{{\em #1}}
}

\reservedcharsoff
\reservedcharson


% ::::::::::::::::::::::::::::::::::::::::::::::::::::

\def\vs{{\em vs.}}
\def\p{\phantom{(0)}}

% Section levels:
\setcounter{secnumdepth}{5}
%  \section{}       % level 1
%  \subsection{}    % level 2
%  \subsubsection{} % level 3
%  \paragraph{}     % level 4 - equivalent to subsubsubsection
%  \subparagraph{}  % level 5

% To show in the table of content:
\setcounter{tocdepth}{5}

% Linebreak after \paragraph
\makeatletter
\renewcommand\paragraph{%
   \@startsection{paragraph}{4}{0mm}%
      {-\baselineskip}%
      {.5\baselineskip}%
      {\normalfont\normalsize\bfseries}}
\makeatother

% Linebreak after \paragraph
\makeatletter
\renewcommand\subparagraph{%
   \@startsection{subparagraph}{4}{0mm}%
      {-\baselineskip}%
      {.5\baselineskip}%
      {\normalfont\normalsize\bfseries}}
\makeatother

\actcharon
\renewcommand{\textfraction}{0.1}
\comment{\paramon\def\herenote#1{}\paramoff}
\renewcommand{\thepage}{\arabic{page}}
\reservedcharson

% :::::::::::: My Additions ::::::::::::::
\newcommand\Spitzer{{\em Spitzer}}
\newcommand\SST{{\em Spitzer Space Telescope}}
\newcommand\chisq{$\chi^2$}
\newcommand\itbf[1]{\textit{\textbf{#1}}}
\newcommand\bftt[1]{\texttt{\textbf{#1}}}
\newcommand\function[1]{\noindent\texttt{\begin{tabular}{@{}l@{}l}#1\end{tabular}}\newline}
\newcommand\bfv[1]{|\textbf{#1}|}
\newcommand\ttred[1]{\textcolor{red}{\ttfamily #1}}
\newcommand\ttblue[1]{\textcolor{blue}{\ttfamily #1}}
\newcommand\ttblack[1]{\textcolor{black}{\ttfamily #1}}
\newcommand\der{{\rm d}}
\newcommand\tno{$\sp{-1}$}
\newcommand\tnt{$\sp{-2}$}
\newcommand*\Eval[3]{\left.#1\right\rvert_{#2}^{#3}}
\newcommand\mcc{MC\sp{3}}

%:::::::::::::::::::::::::::::::::::::::::
% Next six lines adjust spacing above/below captions and Sections etc
% Adjust as needed

\comment{
% \setlength{\abovecaptionskip}{0pt}
% \setlength{\belowcaptionskip}{0pt}
% \setlength{\textfloatsep}{8pt}
% \titlespacing{\section}{0pt}{5pt}{*0}
% \titlespacing{\subsection}{0pt}{5pt}{*0}
% \titlespacing{\subsubsection}{0pt}{5pt}{*0}
}

\reservedcharsoff
\actcharon
\mathshifton

\reservedcharson


\begin{document}

\begin{titlepage}
\begin{center}

\textsc{\LARGE University of Central Florida}\\[1.5cm]

% Title
\rule{\linewidth}{0.5mm} \\[0.4cm]
{ \huge \bfseries LISA Users Manual \\[0.4cm] }
\rule{\linewidth}{0.5mm} \\[1.0cm]

\textsc{\Large Large-selection Interface for Sampling Algorithms}\\[1.5cm]

% Author and supervisor
\noindent
\begin{minipage}{0.4\textwidth}
\begin{flushleft}
\large
\emph{Authors:} \\
Michael D. \textsc{Himes} \\
\end{flushleft}
\end{minipage}%
\begin{minipage}{0.4\textwidth}
\begin{flushright} \large
\emph{Supervisor:} \\
Dr.~Joseph \textsc{Harrington}
\end{flushright}
\end{minipage}
\vfill

% Bottom of the page
{\large \today}

\end{center}
\end{titlepage}

\tableofcontents
\newpage

\section{Team Members}
\label{sec:team}

\begin{itemize}
\item \href{https://github.com/mdhimes/}{Michael Himes}%
  \footnote{https://github.com/mdhimes/}, University of
  Central Florida (mhimes@knights.ucf.edu)
\item Joseph Harrington, University of Central Florida
\end{itemize}

\section{Introduction}
\label{sec:theory}

\noindent This document describes LISA, the Large-selection Interface for 
Sampling Algorithms.  LISA provides a unified interface for different 
Bayesian samplers, both Markov chain Monte Carlo (MCMC) and nested sampling 
(NS).  \newline

\noindent Presently, LISA supports the 
DEMC algorithm of ter Braak (2006), 
DEMCzs of ter Braak \& Vrugt (2008, `snooker'), 
MultiNest of Feroz et al. (2008) via PyMultiNest (Buchner et al. 2014), 
UltraNest of Buchner (2016), 
dynesty of Speagle (2019), and 
DNest4 of Brewer & Foreman-Mackey (2018).  
We welcome users to contribute wrappers for additional 
algorithms via a pull request on Github. \newline

\noindent The detailed LISA code documentation and User Manual\footnote{Most recent version of the manual available at 
\href{https://exosports.github.io/LISA/doc/LISA_User_Manual.html}{https://exosports.github.io/LISA/doc/LISA\_User\_Manual.html}} 
are provided with the package to assist users in its usage. 
For additional support, contact the lead author (see Section \ref{sec:team}). \newline

\noindent LISA is released under the Reproducible Research Software License.  
For details, see \\
\href{https://planets.ucf.edu/resources/reproducible-research/software-license/}{https://planets.ucf.edu/resources/reproducible-research/software-license/}.\newline

\noindent The LISA package is organized as follows: \newline
% The framebox and minipage are necessary because dirtree kills the
% indentation.
\noindent\framebox{\begin{minipage}[t]{0.97\columnwidth}%
\dirtree{%
 .1 LISA. 
 .2 doc.
 .2 example.
 .2 lib. 
 .2 modules. 
 .3 MCcubed. 
}
\end{minipage}}
\vspace{0.7cm}
% \newline is not working here, therefore I use vspace.
% (because dirtree is such a pain in the ass)

\section{Installation}
\label{sec:installation}

\subsection{System Requirements}
\label{sec:requirements}

\noindent LISA was developed on a Linux machine using the following 
versions of packages:

\begin{itemize}
\item Python 3.7.2
\item Numpy 1.16.2
\item Matplotlib 3.0.2
\item Scipy 1.2.1
\item pymultinest 2.10
\item ultranest 2.2.2
\item dynesty 1.0.1
\item DNest4 0.2.4
\end{itemize}


\subsection{Install and Compile}
\label{sec:install}

\noindent To begin, obtain the latest stable version of LISA.  \newline

\noindent First, decide on a local directory to hold LISA.  Let the path to this directory 
be `LISA'.  Now, clone the repository:
\begin{verbatim}
git clone --recursive https://github.com/exosports/LISA LISA/
cd LISA/
\end{verbatim}

\noindent LISA contains a file to easily build a conda environment capable of 
executing the software.  Create the environment via

\begin{verbatim}
conda env create -f environment.yml
\end{verbatim}

\noindent Then, activate the environment:

\begin{verbatim}
conda activate lisa
\end{verbatim}

\noindent Now, build the submodules:

\begin{verbatim}
make all
\end{verbatim}

\noindent Alternatively, `make mc3' accomplishes the same thing.\newline

\noindent You are now ready to run LISA.


\section{Example}
\label{sec:example}

The following script will walk a user through using LISA for a basic quadratic 
fit.

\noindent To begin, copy the requisite files to a directory parallel to LISA. 
Beginning from LISA/, 
\begin{verbatim}
mkdir ../run
cp -a ./example/* ../run/.
cd ../run
\end{verbatim}

\noindent Make your test data.  Execute the provided script with 
desired parameters.  For example, data following \math{3x^2 - x + 2} 
would be created via
\begin{verbatim}
./make_data.py 3 -1 2
\end{verbatim}
\noindent This will produce true.npy, data.npy, uncert.npy, and params.npy, 
which contain the true data, noisy data, uncertainty on each data point, and 
the true parameters, respectively.  The x-axis is assumed to be the integers 
from -5 to 5, inclusive, and the uncertainty is assumed to be the square root 
of the true data.

\noindent Now, execute LISA for each of the samplers:

\begin{verbatim}
./demc_example.py
./dnest4_example.py
./dynesty_example.py
./snooker_example.py
./multinest_example.py
./ultranest_example.py
\end{verbatim}

\noindent Each will produce some output files (logs, plots, etc.) in 
appropriately-named subdirectories.


\section{Program Inputs}
\label{sec:inputs}

LISA's design as an importable package allows for convenient usage in existing 
codes.  Users interact with LISA through its \tt{setup} or \tt{run} functions. 
Each has 1 position argument (the desired algorithm; options are listed below), 
and they take keyword arguments for the different parameters of the sampling 
algorithm.  

\subsection{Sampling Algorithm}

LISA currently supports 6 Bayesian sampling algorithms:
\begin{itemize}
\item demc   : DEMC MCMC algorithm of ter Braak (2006)
\item dnest4 : Diffusive NS implementation of Brewer & Foreman-Mackey (2018)
\item dynesty: Dynamic NS of Speagle (2019)
\item snooker  : DEMCzs MCMC algorithm of ter Braak \& Vrugt (2008)
\item multinest: MultiNest NS algorithm of Feroz et al. (2008)
\item ultranest: UltraNest NS algorithm of Buchner (2014, 2016, 2019).
\end{itemize}

\noindent The choice of sampling algorithm also dictates the allowed inputs.  
Each sampler's inputs are listed in the following subsections.  Bold 
indicates a parameter that must be specified (typically, they have no default); 
optional parameters are given reasonable defaults.  Section 
\ref{sec:param-desc} describes each parameter.  Note that some `optional' 
parameters can strongly influence the results (e.g., dlogz for nested samplers),
so users are encouraged to experiment with them.

\subsubsection{demc \& snooker}
\label{sec:mcmc-inputs}
\begin{itemize}
\item \textbf{burnin}
\item \textbf{data}
\item fbestp
\item fext
\item flog
\item fsavefile
\item fsavemodel
\item hsize (only snooker)
\item indparams
\item kll
\item \textbf{model}
\item modelper
\item \textbf{nchains}
\item \textbf{niter}
\item \textbf{outputdir}
\item \textbf{pinit}
\item \textbf{pmax}
\item \textbf{pmin}
\item pnames
\item \textbf{pstep}
\item thinning
\item truepars
\item \textbf{uncert}
\item verb
\end{itemize}

\subsubsection{dnest4}
\begin{itemize}
\item beta
\item fbestp
\item fext
\item fsavefile
\item kll
\item lam
\item \textbf{loglike}
\item \textbf{model}
\item niter
\item \textbf{nlevel}
\item \textbf{nlevelint}
\item \textbf{nperstep}
\item \textbf{outputdir}
\item \textbf{perturb}
\item pnames
\item \textbf{prior}
\item \textbf{pstep}
\item truepars
\item verb
\end{itemize}

\subsubsection{dynesty, multinest, \& ultranest}
\label{sec:ns-inputs}
\begin{itemize}
\item bound (only dynesty)
\item dlogz
\item fbestp
\item fext
\item frac\_remain (only ultranest)
\item fprefix (only multinest)
\item fsavefile
\item kll
\item Lepsilon (only ultranest)
\item \textbf{loglike}
\item min\_ess (only dynesty \& ultranest)
\item \textbf{model}
\item niter
\item \textbf{nlive}
\item \textbf{nlive\_batch} (only dynesty)
\item \textbf{outputdir}
\item pnames
\item \textbf{prior}
\item \textbf{pstep}
\item sample (only dynesty)
\item truepars
\item verb
\end{itemize}

\subsection{Parameter Dictionary}
\label{sec:param-desc}

Users are only required to specify the relevant parameters listed in 
the previous section.  Each parameter is described below, alphabetically.
To utilize default values, do not include it in **kwargs.

\begin{itemize}
\item beta : float. DNest 4 only. From their docs: strength of effect 
                    to force histogram to equal push.  Default: 100.0
\item bound : str. Dynesty only. Option to bound the target  
                   distribution. Choices: none (sample from unit  
                   cube), single (one ellipsoid), multi (multiple  
                   possibly overlapping ellipsoids), balls  
                   (overlapping balls centered on each live point),  
                   cubes (overlapping cubes centered on each live  
                   point).  Default: multi
\item burnin : int. Number of initial iterations to be discarded.
\item data : array, Numpy binary. Measured data for inference.  
                    Must be Numpy array, list, or a path to a NPY file.
\item dlogz : float. Target evidence uncertainty (stops when below  
                     this value).  Lower = more accurate, but also 
                     requires longer runtime. Default: 0.1
\item fbestp : str. Filename for array of best-fit parameters.  
                    Must be NPY file.
\item fext : str. File extension for saved plots.  
                  Options: .png, .pdf  Default: .png
\item flog : str. MCMCs only. /path/to/log file to save out.  
                  Default: MCMC.log, located in `outputdir`
\item fprefix : str. PyMultiNest only. Prefix for output filenames. 
                     Recommended to be a directory, as this will create 
                     a subdirectory within `outputdir`. Default: pmn/
\item fsavefile : str. Filename to store parameters explored.  
                            If relative path, it is considered with  
                            respect to `outputdir`.   
                            Default: `outputdir`/output.npy
\item fsavemodel : str. MCMCs only (currently). 
                        Filename to store models, corresponding to  
                        the parameters. If relative path, it is  
                        considered with respect to `outputdir`.   
                        If None, file is not saved.   
                        Beware: the file may be extremely large  
                        if the model output has high dimensionality. 
                        See `modelper' to have it automatically split into 
                        subfiles.
                        Default: None
\item frac\_remain : float. UltraNest only. Sets the fraction  
                            remainder when integrating the posterior.
\item hsize : int. Snooker only. Number of samples per chain to seed the 
                   phase space.  Default: 10
\item indparams : list. MCMCs only. Additional parameters needed by `model`.
\item kll : object.  Datasketches KLL object, for model quantiles.  
                     Use None if not desired or if Datasketches is  
                     not installed.  Default: None
\item lam : float. DNest 4 only. From their docs: backtracking scale 
                   length.  Default: 5.0
\item Lepsilon : float. UltraNest only. From their docs: "Terminate  
                        when live point likelihoods are all the same,  
                        within Lepsilon tolerance."
\item loglike : object. Nested samplers only.  Function defining the log 
                        likelihood.
\item min\_ess : int. Minimum effective sample size (ESS). Default: 500
\item model : object. Function defining the forward model.
\item modelper : int. MCMCs only. Sets how to split `fsavemodel` 
                      into subfiles.   
                      If 0, does not split.  If >0, saves out every  
                      `modelper` iterations.  E.g., if nchains=10 and  
                      modelper=5, splits every 50 model evaluations.   
                      Default: 0
\item nchains : int. Number of parallel samplers. Default: 1
\item niter : int. Maximum number of iterations.  Nested samplers  
                       default to no limit.
\item nlive : int. (Minimum) number of live points to use. Default: 500
\item nlive\_batch: int. Dynesty only. From their docs: "The number of  
                         live points used when adding additional  
                         samples from a nested sampling run within  
                         each batch."  Default: 500.
\item nlevel : int. DNest4 only. From their docs: Maximum number of levels to 
                    create.  Default: 30
\item nlevelint : int. DNest4 only. Number of moves before creating new level. 
                       Default: 10000
\item nperstep : int. DNest4 only. Number of moves per MCMC iteration.  
                      Default: 10000
\item outputdir : str. path/to/directory where output will be saved.
\item perturb : object. DNest4 only. Function that proposes changes to 
                        parameter values.
\item pinit : array, Numpy binary. For MCMCs, initial parameters for  
                            samplers.  For nested sampling algorithms,  
                            values are used for parameters that are held  
                            constant, if any. 
                        Must be Numpy array, list, or a path to a NPY file.
\item pmax : array, Numpy binary. Maximum value for each parameter. 
                        Must be Numpy array, list, or a path to a NPY file.
\item pmin : array, Numpy binary. Minimum value for each parameter. 
                        Must be Numpy array, list, or a path to a NPY file.
\item pnames : array. Name of each parameter (can use some LaTeX if  
                          desired). 
                          Must be Numpy array or list.
\item prior : object. Function defining the prior.
\item pstep : array. Step size for each parameter.  For MCMCs, only  
                         matters for the initial samples and determining  
                         constant parameters, as step size is  
                         automatically adjusted.  For nested sampling  
                         algorithms, only used to determine constant  
                         parameters.
\item sample : str. Dynesty only. Sampling method. Choices  
                        (descriptions from their docs): unif (uniform  
                        sampling), rwalk (random walks from current live  
                        point), rstagger (random "staggering" away from  
                        current live point), slice (multivariate slice  
                        sampling), rslice (random slice sampling), hslice  
                        ("Hamiltonian" slice sampling), auto  
                        (automatically selected based on problem  
                        dimensionality).  Default: auto
\item thinning : int. Thinning factor for the posterior  
                      (keep every N iterations).  
                      Example: a thinning factor of 3 will keep every  
                      third iteration.  Only recommended when the  
                      computed posterior is extremely large.   
                      Default: 1
\item truepars : array, Numpy binary. Known true values for the model  
                          parameters.  If unknown, use None.  Default: None
\item uncert : array, Numpy binary. Data uncertainties. 
                        Must be Numpy array, list, or a path to a NPY file.
\item verb : int. Verbosity level.  If 0, only essential messages 
                       are printed by LISA.  If 1, prints additional 
                       messages (e.g., loading data files).  
                       In the future, additional levels may be added.  
                       Default: 0
\end{itemize}


\section{Program Outputs}
\label{sec:outputs}

LISA has two main functions: \tt{setup} and \tt{run}.  \tt{setup} returns 
an initialized Sampler object.  \tt{run} returns the Sampler object (with 
additional attributes: outp for the posterior and bestp for the best parameters) and produces at least 3 plots.

\subsection{Returns}
\begin{itemize}
\item samp: Sampler object. Contains the attributes listed Sections 
\ref{sec:mcmc-inputs} -- \ref{sec:ns-inputs}.
\end{itemize}

\subsection{Output Files}
\begin{itemize}
\item pairwise: corner plot of histograms of the 2D marginalized posteriors.
\item posterior: histogram plots of the 1D marginalized posteriors.
\item trace: parameter history plots.
\end{itemize}

\noindent If fsavefile and fbestp are not None, there are two NPY files 
produced.
\begin{itemize}
\item bestp.npy: Array of best parameters.
\item outp.npy: Array of the approximation to the posterior.
\end{itemize}

\noindent Each sampler also has additional output files, listed below.

TODO fill in rest of the outputs for each method

\section{FAQ}

This section will cover some frequently asked questions about using LISA.  
Have a question that isn't answered below?  Please send it to the corresponding 
author so we can add it to this section! \newline

\noindent \textbf{Q: What is effective sample size (ESS) and why should I care?}
A: As Bayesian samplers explore a phase space, the subsequent sets of parameters
are based in part on earlier sets of parameters.  Thus, iterations are 
correlated.  After some number of samples, new iterations will ``forget'' some 
of the earlier samples.  Yet, we are really interested in the number of 
independent samples, as that will inform how thoroughly the phase space has 
been explored (taking a bunch of correlated samples doesn't give us much new 
information!).  The number of iterations necessary before a new sample is 
independent of a given earlier sample (steps per effective independent sample, 
SPEIS) informs the ESS.  For example, if we have 20,000 samples, but a given 
sample is correlated with the previous 100 samples, then the ESS is just 
20000 / 100, or 200.  The ESS value, in turn, informs us of the accuracy of 
a credible region determined from the posterior distribution.  Without 
providing a detailed proof (interested readers can find it in Harrington et 
al. 2021, Appendix A), the relationship between the ESS, a credible region 
\math{\hat{C}}, and the uncertainty on the credible region 
\math{s\sb{\hat{C}}} is
\begin{equation}
{\rm ESS} \approx \frac{\hat{C}(1-\hat{C})}{s\sbp{\hat{C}}{2}}.
\end{equation}
Thus, for a 0.5\% uncertainty on the 95.45\% region (2\math{\sigma} for a 
Gaussian), an ESS \math{\approx} 1700 is required.  A SPEIS of 1000 would 
thus require 1.7 million iterations!


\section{Be Kind}
\label{sec:bekind}
Please note that this software has not been officially released yet; see the 
license for some restrictions on its usage prior to release.  Upon release, 
we will add the relevant citation here, with a Bibtex entry.

\noindent Thanks!

% \section{Further Reading}
% \label{sec:furtherreading}

% TBD: Add papers here.


\end{document}
