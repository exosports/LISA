% LISA User Manual
%
% Please note this document will be automatically compiled and hosted online
% after each commit to master. Because of this, renaming or moving the
% document should be done carefully. To see the compiled document, go to
% https://exosports.github.io/LISA/doc/LISA_User_Manual.html

\documentclass[letterpaper, 12pt]{article}
\input{top-LISA_user_manual}

\begin{document}

\begin{titlepage}
\begin{center}

\textsc{\LARGE University of Central Florida}\\[1.5cm]

% Title
\rule{\linewidth}{0.5mm} \\[0.4cm]
{ \huge \bfseries LISA Users Manual \\[0.4cm] }
\rule{\linewidth}{0.5mm} \\[1.0cm]

\textsc{\Large Large-selection Interface for Sampling Algorithms}\\[1.5cm]

% Author and supervisor
\noindent
\begin{minipage}{0.4\textwidth}
\begin{flushleft}
\large
\emph{Authors:} \\
Michael D. \textsc{Himes} \\
\end{flushleft}
\end{minipage}%
\begin{minipage}{0.4\textwidth}
\begin{flushright} \large
\emph{Supervisor:} \\
Dr.~Joseph \textsc{Harrington}
\end{flushright}
\end{minipage}
\vfill

% Bottom of the page
{\large \today}

\end{center}
\end{titlepage}

\tableofcontents
\newpage

\section{Team Members}
\label{sec:team}

\begin{itemize}
\item \href{https://github.com/mdhimes/}{Michael Himes}%
  \footnote{https://github.com/mdhimes/}, University of
  Central Florida (mhimes@knights.ucf.edu)
\item Joseph Harrington, University of Central Florida
\end{itemize}

\section{Introduction}
\label{sec:theory}

\noindent This document describes LISA, the Large-selection Interface for 
Sampling Algorithms.  LISA provides a unified interface for different 
Bayesian samplers, both Markov chain Monte Carlo (MCMC) and nested sampling 
(NS).  

Presently, LISA supports the DEMC algorithm of ter Braak (2006), DEMCzs of 
ter Braak \& Vrugt (2008), MultiNest of Feroz et al. (2008), and UltraNest of 
Buchner (2016).  We welcome users to contribute interfaces for additional 
algorithms via a pull request on Github.

The detailed LISA code documentation and User Manual\footnote{Most recent version of the manual available at 
\href{https://exosports.github.io/LISA/doc/LISA_User_Manual.html}{https://exosports.github.io/LISA/doc/LISA\_User\_Manual.html}} 
are provided with the package to assist users in its usage. 
For additional support, contact the lead author (see Section \ref{sec:team}).

LISA is released under the Reproducible Research Software License.  
For details, see \\
\href{https://planets.ucf.edu/resources/reproducible-research/software-license/}{https://planets.ucf.edu/resources/reproducible-research/software-license/}.
\newline

\noindent The LISA package is organized as follows: \newline
% The framebox and minipage are necessary because dirtree kills the
% indentation.
\noindent\framebox{\begin{minipage}[t]{0.97\columnwidth}%
\dirtree{%
 .1 LISA. 
 .2 doc.
 .2 example.
 .2 lib. 
 .2 modules. 
 .3 MCcubed. 
}
\end{minipage}}
\vspace{0.7cm}
% \newline is not working here, therefore I use vspace.
% (because dirtree is such a pain in the ass)

\section{Installation}
\label{sec:installation}

\subsection{System Requirements}
\label{sec:requirements}

\noindent LISA was developed on a Linux machine using the following 
versions of packages:

\begin{itemize}
\item Python 3.7.2
\item Numpy 1.16.2
\item Matplotlib 3.0.2
\item Scipy 1.2.1
\end{itemize}

\noindent It utilizes the most up-to-date versions of MCcubed (submodule), 
PyMultiNest, and UltraNest.


\subsection{Install and Compile}
\label{sec:install}

\noindent To begin, obtain the latest stable version of LISA.  

\noindent First, decide on a local directory to hold LISA.  Let the path to this directory 
be `LISA'.  Now, clone the repository:
\begin{verbatim}
git clone --recursive https://github.com/exosports/LISA LISA/
cd LISA/
\end{verbatim}

\noindent LISA contains a file to easily build a conda environment capable of 
executing the software.  Create the environment via

\begin{verbatim}
conda env create -f environment.yml
\end{verbatim}

\noindent Then, activate the environment:

\begin{verbatim}
conda activate lisa
\end{verbatim}

\noindent Now, build the submodules:

\begin{verbatim}
make all
\end{verbatim}

\noindent Alternatively, `make mc3' accomplishes the same thing.

\noindent You are now ready to run LISA.


\section{Example}
\label{sec:example}

The following script will walk a user through using LISA for a basic quadratic 
fit.

\noindent To begin, copy the requisite files to a directory parallel to LISA. 
Beginning from LISA/, 
\begin{verbatim}
mkdir ../run
cp -a ./example/* ../run/.
cd ../run
\end{verbatim}

\noindent Make your test data.  Execute the provided script with 
desired parameters.  For example, data following \math{3x^2 - x + 2} 
would be created via
\begin{verbatim}
./make_data.py 3 -1 2
\end{verbatim}
\noindent This will produce true.npy, data.npy, uncert.npy, and params.npy, 
which contain the true data, noisy data, uncertainty on each data point, and 
the true parameters, respectively.  The x-axis is assumed to be the integers 
from -5 to 5, inclusive, and the uncertainty is assumed to be the square root 
of the true data.

\noindent Now, execute LISA for each of the samplers:

\begin{verbatim}
./demc_example.py
./snooker_example.py
./multinest_example.py
./ultranest_example.py
\end{verbatim}

\noindent Each will produce some output files (logs, plots, etc.) in appropriately-named 
subdirectories.


\section{Program Inputs}
\label{sec:inputs}

LISA's design as an importable package allows for convenient usage in existing 
codes.  Users interact with LISA through its \tt{run} function, which takes two 
arguments: a string of the sampling algorithm, and a dictionary of parameters.

\subsection{Sampling Algorithm}

LISA currently supports 4 Bayesian sampling algorithms:
\begin{itemize}
\item demc   : DEMC MCMC algorithm of ter Braak (2006)
\item snooker  : DEMCzs MCMC algorithm of ter Braak \& Vrugt (2008)
\item multinest: MultiNest NS algorithm of Feroz et al. (2008)
\item ultranest: UltraNest NS algorithm of Buchner (2014, 2016, 2019).
\end{itemize}

\noindent The choice of sampling algorithm also dictates which entries 
are used from the parameter dictionary.  Each sampler's inputs are 
listed in the following subsections.

\subsubsection{DEMC \& DEMCzs}
\begin{itemize}
\item burnin
\item data
\item flog
\item fsavefile
\item fsavemodel
\item func
\item hsize (only DEMCzs)
\item indparams
\item nchains
\item niter
\item outputdir
\item pinit
\item pmax
\item pmin
\item pnames
\item pstep
\item savefile
\item thinning
\item truepars
\item uncert
\end{itemize}

\subsubsection{MultiNest \& UltraNest}
\begin{itemize}
\item kll
\item loglike
\item model
\item outputdir
\item pnames
\item prior
\item pstep
\item savefile
\end{itemize}


\subsection{Parameter Dictionary}

Users are only required to specify the relevant parameters listed in 
the previous section.  Each parameter is described below, alphabetically.
To utilize default values, do not include that key in the dictionary.

\begin{itemize}
\item burnin: int. Number of burned iterations.
\item data: array, Numpy binary. Measured data for inference. Must be Numpy 
                                 array or a path to a .NPY file.
\item flog: str.   /path/to/log file.
\item fsavefile: str. Filename to store parameters explored.
\item fsavemodel: str. Filename to store models, corresponding to the parameters.
\item func: object. Function to be evaluated at each MCMC step.
\item hsize: int. Number of samples to seed the phase space (DEMCzs only).
\item indparams: list. Additional parameters needed by `func`.
\item kll: object.  Datasketches KLL object, for model quantiles.
                    Use None if not desired or if Datasketches is not installed.
                    Default: None
\item loglike: object. Function defining the log likelihood.
\item model: object. Function defining the forward model.
\item nchains: int. Number of parallel samplers. Default: 1
\item niter: int. Total number of iterations.
\item outputdir: str. path/to/directory where output will be saved.
\item pinit: array. Initial parameters for MCMC.
\item pmax: array. Maximum value for each model parameter.
\item pmin: array. Minimum value for each model parameter.
\item pnames: array. Name of each parameter (can use some LaTeX if desired).
\item pstep: array. Step size for each parameter.  For DEMC \& variants, only 
                    matters for the initial samples, as step size is automatically adjusted.
                    For NS, only used to determine any constant parameters.
\item savefile: str. Prefix for the plots that will be saved. Default: empty string
\item thinning: int. Thinning factor for the posterior. Default: 1
\item truepars: array. Known true values for the model parameters.  
                       If unknown, use None.  Default: None
\item uncert: array, Numpy binary. Assumed uncertainties for inference. Must be 
                                   Numpy array or a path to a .NPY file.
\end{itemize}


\section{Program Outputs}
\label{sec:outputs}

LISA produces different output files depending on the chosen algorithm.
Regardless of algorithm, LISA returns 2 arrays and produces 3 plots.

\subsection{Returns}
\begin{itemize}
\item outp: array. Output posterior.
\item bestp: array. Best-fit parameters.
\end{itemize}

\subsection{Output Files}
\begin{itemize}
\item TODO
\end{itemize}


\section{Be Kind}
\label{sec:bekind}
Please cite this paper if you found this package useful for your
research:

\begin{itemize}
\item Himes et al. (2020), submitted to PSJ.
\end{itemize}

\begin{verbatim}
@article{HimesEtal2020psjMARGEHOMER,
   author = {{Himes}, Michael D. and {Harrington}, Joseph and {Cobb}, Adam D. and {G{\"u}ne{\textcommabelow s} Baydin}, At{\i}l{\i}m and {Soboczenski}, Frank and
         {O'Beirne}, Molly D. and {Zorzan}, Simone and
         {Wright}, David C. and {Scheffer}, Zacchaeus and
         {Domagal-Goldman}, Shawn D. and {Arney}, Giada N.},
    title = "Accurate Machine Learning Atmospheric Retrieval via a Neural Network Surrogate Model for Radiative Transfer",
  journal = {PSJ},
     year = 2020,
    pages = {submitted to PSJ}
}
\end{verbatim}

\noindent Thanks!

% \section{Further Reading}
% \label{sec:furtherreading}

% TBD: Add papers here.


\end{document}
